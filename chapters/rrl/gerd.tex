Competitive Programming (CP) is a sport that involves programmers competing 
against each other to solve programming problems with a time limit. The programmers 
are then evaluated based on the factors such as efficiency, time complexity, 
and the accuracy of the code.\cite{majumdar2017macacm} This paradigm was based on an 
event in Bulgaria 30 years ago called the International Olympiad in Information (IOI). 
Nowadays, the majority of these competitions are done entirely online using platforms 
such as CodeChef, HackerRank, HackerEarth, and others. It can have contestants ranging 
from two to several thousand and span topics such as data structures \& algorithms, data science, and discrete mathematics\cite{nair2020increasing}.

\section{Competitive Programming as Pedagogy}

The National University of Singapore, University of West Indies, 
and other universities have been using CP as an alternative method 
of teaching students by utilizing it as a module in replacement to intermediate-level data structures and algorithms subjects\cite{coore2019facilitating}. 
Training programs\cite{di2018learning} and teaching frameworks\cite{di2018framework} 
were also created revolving around CP as the major learning material which uses Association 
for Computing Machinery International Collegiate Programming Contest (ACM-ICPC) and IOI 
style of evaluating the performance of the students.  Furthermore, employers and graduates 
are utilizing CP as a preparatory material to adapt to the competitive environment of the 
software industry.\cite{nair2020increasing} Nair states that in India, many people joining the IT 
workforce are unemployable and use CP to increase employability and  introducing CP not 
only improves the skills of these students but also dramatically enhances their capability  of collaboration and creativity.