%%%%%%%%%%%%%%%%%%%%%%%%%%%%%%%%%%%%%%%%%%%%%%%%%%%%%%%%%%%%%%%%%%%%%%%%%%%
\chapter{Methodology}
%%%%%%%%%%%%%%%%%%%%%%%%%%%%%%%%%%%%%%%%%%%%%%%%%%%%%%%%%%%%%%%%%%%%%%%%%%%



\section{Dataset}

The researchers will be using a custom dataset mined from public GitHub 
repositories that contain Python, JavaScript, and Java source code, such as 
those repositories that implement  algorithms and data structures. Labels 
will be extracted from a number of heuristics such as method name or the 
repository folder in the case of algorithm implementation datasets. Since 
CodeBERT only accepts input lengths of less than 512, it is important to 
preprocess the dataset to make sure that each code snippet contains less than 
512 tokens\cite{devlin2018bert}\cite{liu2019roberta}.

\textbf{RQ1.} Data Augmentation\cite{feng2021survey}
will be done to make the dataset synthetically larger. To counter the challenge of 
overfitting, that is, making sure that the model will not simply memorize variable 
names, the researchers will be using synonym replacements on identifiers and swapping 
consecutive lines that are declaring variables as these do not have an effect on the code. 
Custom code extraction programs will be written for each programming language using available parsers.

\textbf{RQ2.} Code snippets should be able to preserve coding styles except for the 
comments which will not be considered.

\section{Training}

CodeBERT will be used for transfer learning on the custom dataset using a multiclass 
classification task with each one-hot encoded label corresponding to one algorithm. 
Different hyperparameters on the output layer will be used for model selection. The 
researchers will make use of the PyTorch\cite{paszke2019pytorch} library for implementation 
and PyTorch Lightning for training\cite{Falcon_PyTorch_Lightning_2019}.

\section{Model Selection}

The best performing model will be selected with the criteria based on its F1-score.
Precision and recall are metrics that are calculated based on the true positives, 
false positives, true negatives, and false negatives given by the formulae. 
Once the best model has been selected, a prototype implementation of Project 
CodeC’s task-constrained feedback feature will be implemented to test on 
actual programming competition datasets. \( precision = \frac{true positive}{true positive + false positive}\) and 
\( recall = \frac{true positive}{true positive + false negative}\). The F1 score balances these metrics using the formula 
\( F1 = \frac{2(precision)(recall)}{precision + recall}\).\newline

\textbf{RQ1.} The model with the highest F1 score must be the one with the best performing model. 
Other performance metrics may be collected such as running loss and classification accuracy.

\textbf{RQ2.} By recording the examples that resulted in false positives and false negatives, manual analysis will 
be performed to identify the features that the transformer was not able to fully capture.


