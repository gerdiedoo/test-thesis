%%%%%%%%%%%%%%%%%%%%%%%%%%%%%%%%%%%%%%%%%%%%%%%%%%%%%%%%%%%%%%%%%%%%%%%%%%%
\chapter{Contributions and Recommendations}
%%%%%%%%%%%%%%%%%%%%%%%%%%%%%%%%%%%%%%%%%%%%%%%%%%%%%%%%%%%%%%%%%%%%%%%%%%%

\section{Summary of Contributions}


\section{Recommendations}


There could be several reasons why our tests did not yield the ``same'' results as that of \cite{Zaritsky2004}'s.
The author claims that this may be due to the myriad of parameters used in evolving
individuals in each step of the genetic algorithm and the large number of possible candidates or values that we can set these parameters with. 
For example, we may claim that the appropriate choice of the recombination technique is necessary. In \cite{Zaritsky2004},
a two-point crossover was employed. Thinking that this may be less of a factor, the author proceeded with just
one-point cross in implementing the genetic algorithm for the experiment.
Another could be in choice of the probabilities associated with recombination, elitism, and mutation rates.
The author did only consider what was stated in the base paper without performing tests to validate
whether this values are appropriate and could still complement the change made in some of the assumptions.
As such, more empirical tests could be done to appropriately estimate these parameters. 

If indeed DNA sequence assembly is the task at hand, lots of other techniques are now available.
One may consider for example representing the reads or DNA fragments in a de Bruijn graph
as opposed to the overlap graph presented in this paper. 
Assembling reads  in a de Bruijn graph 
reduces the problem to a fragment assembly problem that can be 
formulated as the goal to find a trail or Eularian path that visits 
each edge (read or fragment) in the (de Bruijn) graph exactly once. 
Such, somehow makes the assembly process much ``easier'' since Eularian path
construction is known to be solvable in deterministic polynomial time.

